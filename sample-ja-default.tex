\documentclass[xelatex,a4paper,ja=standard]{bxjsarticle}
% \usepackage[haranoaji,threeweight]{zxjafont}
\usepackage{metalogo}
\usepackage{amsmath}
\title{\XeLaTeX による日本語ドキュメント例 (デフォルト)}
\subtitle{サブタイトル?}
\author{Shin Saito}

\newcommand{\jasample}{武士道はそのシンボルである桜花と等しく、日本の地に固有の花である。}
\newcommand{\ensample}{The quick brown fox jumped over the lazy dog.}

\begin{document}
\maketitle

以下、デフォルト設定、ここでは \texttt{bxjsarticle} クラスに \texttt{ja=standard} だけ指定した場合の振る舞いである。
この場合、新しい TeXLive では和文に原ノ味フォントが使われているようであるが、マルチウェイトには対応していないようで、ライトウェイトや極太ウェイトを指定することはできない。

\section{フォントファミリによる違い}

\subsection{セリフ体 (\texttt{\textbackslash rmfamily})}

デフォルト設定では \texttt{\textbackslash ltseries} でライトウェイトを指定することはできない。
また、和文のボールドウェイトはゴシック体で表示される。

\rmfamily\LARGE

% {\ltseries (ライト) 日本語: \jasample

% English: the quick brown fox jumped over the lazy dog.}

{\mdseries (標準) 日本語: \jasample

English: the quick brown fox jumped over the lazy dog.}

{\bfseries (太字) 日本語: \jasample

English: the quick brown fox jumped over the lazy dog.}

{\mdseries リガチャのテスト: file, flow, off, difficult, offline, off\/line, off\textcompwordmark line.}
 
\normalsize
\begin{gather*}
    S = \sum_{i=1}^n i \\
    \Gamma \vdash x \colon \tau \quad \Gamma \vdash {\neg(A\land B)} \iff {\neg A} \lor {\neg B}
\end{gather*}

\subsection{サンセリフ体 (\texttt{\textbackslash sffamily})}

サンセリフ体では \texttt{\textbackslash ebseries} で極太ウェイトが指定できるようになる。
こちらも英文フォントに対しては適用されない。

\sffamily\LARGE

{\mdseries (標準) 日本語: \jasample

English: the quick brown fox jumped over the lazy dog.}

{\bfseries (太字) 日本語: \jasample

English: the quick brown fox jumped over the lazy dog.}

% {\ebseries (極太) 日本語: \jasample

% English: the quick brown fox jumped over the lazy dog.}

{\mdseries リガチャのテスト: file, flow, off, difficult, offline, off\/line, off\textcompwordmark line.}

\normalsize
\begin{gather*}
    S = \sum_{i=1}^n i \\
    \Gamma \vdash x \colon \tau \quad \Gamma \vdash {\neg(A\land B)} \iff {\neg A} \lor {\neg B}
\end{gather*}

\subsection{タイプライタ体 (\texttt{\textbackslash ttfamily})}

\ttfamily\LARGE

{\mdseries (標準) 日本語: \jasample

English: the quick brown fox jumped over the lazy dog.}

{\bfseries (太字) 日本語: \jasample

English: the quick brown fox jumped over the lazy dog.}

\normalsize
(等幅フォントにはリガチャがないので省略)

\begin{gather*}
    S = \sum_{i=1}^n i \\
    \Gamma \vdash x \colon \tau \quad \Gamma \vdash {\neg(A\land B)} \iff {\neg A} \lor {\neg B}
\end{gather*}

\end{document}